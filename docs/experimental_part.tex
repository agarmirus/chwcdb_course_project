\chapter{Исследовательская часть}

Целью исследования является определение зависимости среднего времени получения результата запроса на стороне фронтенда от параметра TTL кэша.

Исследование было проведено на электронной вычислительной машине, обладающей следующими характеристиками:
\begin{itemize}
	\item операционная система Manjaro Linux x86\_64~\cite{manjaro};
	\item процессор Intel i7-10510U 4.900 ГГц~\cite{cpu};
	\item оперативная память DDR4, 2400 МГц, 8 ГБ~\cite{ram}.
\end{itemize}

Для измерения процессорного времени была использована <данные~удалены>~\cite{}. Для построения графиков был использован пакет PGFPlots~\cite{pgfplots}.

В таблице~\ref{ttl_time_table} приведены результаты измерений среднего времени получения результата запроса на стороне фронтенда от параметра TTL кэша. В процессе исследования программе подавалось 10000 запросов с интервалом в 1 секунду.
\begin{center}
	\begin{threeparttable}
		\captionsetup{justification=raggedright,singlelinecheck=off}
		\caption{\label{ttl_time_table}Результаты измерений среднего времени получения результата запроса на стороне фронтенда от параметра TTL кэша}
		\centering
		\begin{tabular}{|c|c|}
			\hline
			TTL, мс & Среднее время получения результата запроса, нс \\
			\hline
		\end{tabular}
	\end{threeparttable}
\end{center}

На рисунке~\ref{ttl_time_graph} приведен график зависимости среднего времени получения результата запроса на стороне фронтенда от параметра TTL кэша.

<данные~удалены>

\section{Вывод}

<данные-удалены>

\clearpage
