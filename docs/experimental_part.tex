\chapter{Исследовательская часть}

Целью исследования является определение зависимости среднего времени получения результата запроса от параметра TTL кэша.

Исследование было проведено на электронной вычислительной машине, обладающей следующими характеристиками:
\begin{itemize}
	\item операционная система Manjaro Linux x86\_64~\cite{manjaro};
	\item процессор Intel i7-10510U 4.900 ГГц~\cite{cpu};
	\item оперативная память DDR4, 2400 МГц, 8 ГБ~\cite{ram}.
\end{itemize}

Для измерения процессорного времени была использована <данные~удалены>~\cite{}. Для построения графиков был использован пакет PGFPlots~\cite{pgfplots}.

В таблице~\ref{ttl_time_table} приведены результаты измерений среднего времени получения результата select-запроса от параметра TTL кэша. В процессе исследования программе подавалось 10000 запросов с интервалом в 50 микросекунд.
\begin{center}
	\begin{threeparttable}
		\captionsetup{justification=raggedright,singlelinecheck=off}
		\caption{\label{ttl_time_table}Результаты измерений среднего времени получения результата select-запроса от параметра TTL кэша}
		\centering
		\begin{tabular}{|c|c|}
			\hline
			TTL, мс & Среднее время получения результата запроса, нс \\
			\hline
			1 & 1349818 \\
			\hline
			10 & 1301967 \\
			\hline
			50 & 1230938 \\
			\hline
			100 & 1200040 \\
			\hline
			500 & 1192420 \\
			\hline
			1000 & 1206887 \\
			\hline
			2500 & 1199234 \\
			\hline
			5000 & 1201852 \\
			\hline
			7500 & 1202476 \\
			\hline
			10000 & 1194852 \\
			\hline
		\end{tabular}
	\end{threeparttable}
\end{center}

На рисунке~\ref{ttl_time_graph} приведен график зависимости среднего времени получения результата select-запроса от параметра TTL кэша.
\begin{figure}[H]
	\begin{center}
		\begin{tikzpicture}
			\begin{axis}[
				title style={align=center},
				xlabel = {TTL, мс},
				ylabel = {Срдене время получения результата, нс},
				legend pos = north east,
				legend style={font=\tiny},
				width = \linewidth
				]
				\legend{ 
					Среднее время получения результата
				};
				\addplot[mark=square*, color=blue] coordinates
				{
					(1, 1339818)
					(10, 1301967)
					(50, 1230938)
					(100, 1200040)
					(500, 1192420)
					(1000, 1206887)
					(2500, 1199234)
					(5000, 1201852)
					(7500, 1202476)
					(10000, 1194852)
				};
			\end{axis}
		\end{tikzpicture}
		\caption{\label{ttl_time_graph}График зависимости среднего времени получения результата запроса от параметра TTL кэша}
	\end{center}
\end{figure}

\section{Вывод}

В исследовательской части была определена зависимость среднего времени получения результата запроса от параметра TTL кэша.
Из рисунка~\ref{ttl_time_graph} можно сделать вывод, что время получения результата запроса уменьшается по мере увеличения параметра TTL кэша, при этом при TTL большем 1000 мс наблюдается насыщение.
Это связано с тем, что данные не успевают удалиться из кэша до завершения выполнения запросов.

\clearpage
