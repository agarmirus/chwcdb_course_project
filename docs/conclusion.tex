\ssr{ЗАКЛЮЧЕНИЕ}

В рамках курсовой работы была разработана база данных сыгранных на кубке мира шахматных партий и приложение к ней.

Был проведен анализ предметной области, связанной с проведением кубка мира по шахматам и ставками на спорт. Были рассмотрены и сравнены существующие решения для хранения шахматных партий. Были сформулированы требования к проектируемым программному обеспечению и базе данных. Были рассмотрены системы управления базами данных на основе формализованной задачи. Были описаны сущности проектируемой базы данных и пользователи разрабатываемого приложения.

Были формализованы бизнес-правила приложения и спроектирована база данных. Были описаны ролевая модель и ограничения базы данных. Были разработаны схемы алгоритмов триггеров, необходимых для корректной работы системы. Была описана структура разрабатываемого приложения.

Были проанализированы и выбраны средства реализации приложения и базы данных. Были описаны триггеры, пользователи и ограничения целостности базы данных. Был разработан графический пользовательский интерфейс приложения.

Было проведено исследование, целью которого являлось определение зависимости среднего времени получения результата запроса на стороне фронтенда от параметра TTL кэша. По результатам измерений можно сделать вывод, что <данные~удалены>.

Были решены следующие задачи:
\begin{itemize}
	\item проведение анализа предметной области, связанной с шахматными турнирами;
	\item формулировка требований к базе данных и приложению;
	\item описание пользователей проектируемого приложения;
	\item проектирование архитектуры базы данных и ограничений целостности;
	\item проектирование ролевой модели на уровне базы данных;
	\item анализ и выбор средств реализации базы данных и приложения;
	\item реализация спроектированной базы данных и необходимого интерфейса для взаимодействия с ней;
	\item исследование характеристик разработанного программного обеспечения.
\end{itemize}
